\section{Introduction}
\label{sec:intro}

Researchers around the world are linking more and more data to ontologies which are formal specification of concepts and relationships in various domains. Formal ontologies have been extensively developed and harnessed in scientific research, particularly in biomedical research. Knowledge evolves rapidly in biomedicine and has promoted the creation and use of ontologies to advance scientific progress.  Besides the size of data increases exponentially, an increasing number of ``large" biomedical ontologies are developed. Prominent examples of this effort include the Gene Ontology (GO)~\cite{GO} and the Unified Medical Language System (UMLS)~\cite{UMLS}. Over 300 ontologies have been loaded into the National Center of Biomedical Ontology (NCBO) BioPortal library at Stanford~\cite{Noy2009}, specifying more than 5.6 million terms in the biomedical domain.

There are two major challenges facing researchers when it comes to mining large sets of biomedical ontologies and data. The first is to leverage both ontologies and data in a systematic and scalable way. The second is to deal with errors in both ontologies and data since neither of them is perfect in reality. Previous research has not been sufficient to address these challenges. For example, some approaches have utilized ontologies in data mining, but usually only a small portion of ontologies is used (typically the subsumption relationship) on very few tasks (most often concept aggregation). On the other hand, limited attempts have been made to check errors in large ontologies. Traditional approaches to sanitize knowledge bases by using an inference engine for logic reasoning and consistency checking is hard to scale.

With the increasing amount of ontology-annotated data, new possibilities are opened up for both data mining and ontology development. Therefore an emerging research direction, which we call \textit{semantic data mining}, focuses on drawing insights from both domain knowledge and data in a systematic way. It aims at bringing domain knowledge seamlessly into the data mining process, and helps improve the quality of pattern discovered in a noisy environment. It also benefits the ontologies by utilizing empirical substantiation from data to either bolster a priori ontological assertions, or detect potential errors therein.

Semantic data mining leverages links between entities defined by ontologies---via annotations---to the mining algorithms explicitly in a unified model. This requires traversing links across the ontologies to infer implicit inter-connections among the data. Graph techniques fit this research nicely because both domain knowledge and semantically rich datasets can be represented as graphs. For example, OWL~\cite{OWL} is the standard ontology language built on RDF. Inheriting the graph nature of RDF, any collection of OWL ontologies or RDF data is an RDF Graph~\cite{GraphModelRDF}. In fact, many semantically rich datasets of interest today, such as DBpedia,  are best described as a linked collection, or a graph, of interrelated objects~\cite{LinkMiningGetoor}. These graphs may represent both homogeneous networks and heterogeneous networks.

Hence, our semantic data mining approach is inspired by a combination of graph representation~\cite{CheinMugnier08}, hypergraphs~\cite{Zhou06learningwith}, and random walks~\cite{Fouss06random-walkcomputation, Zhou:2009:GCB:1687627.1687709}. This paper extends our previous work that implements a hypergraph-based approach to learn associations from interlinked data (without ontologies)~\cite{LiuEtal11}. The RDF hypergraph representation proposed by Hayes et al.~\cite{GraphModelRDF} is a key innovation which connects data and ontologies. We adopt this representation in our approach because properties in RDF triples can be represented as first class objects among all interrelated objects, enabling us to embed both the ontology and the data together and to serialize it into a bipartite graph representation for scalable processing.

In RDF hypergraphs, we use random walk with restart to efficiently calculate the similarity between concepts to determine their associations from large set of data and ontologies. Traditional association mining relies on co-frequencies of items (concepts) within transactions~\cite{Agrawal94}. We look one step further to find indirectly associated items (concepts). This extension has far reaching applications in biomedicine. For instance, consider a simple scenario illustrated by Swanson~\cite{swanson87} years ago while studying Raynauld's syndrome. He noticed that the literature discussed Raynauld's syndrome ($Z$), a peripheral circulatory disease, together with certain changes of blood in human body ($Y$); and, separately, the consumption of dietary fish oil ($X$) was also linked in the literature to similar blood changes.  But fish oil and Raynauld's syndrome were never linked directly in any previous publications.  Swanson reasoned (correctly) that fish oil could potentially be used to treat Raynauld's syndrome, i.e., $X\rightsquigarrow Y \rightsquigarrow Z$. We call such indirect associations, $(X,Z)$, \emph{semantic associations}.

We evaluate the effectiveness of the results on large real-world biomedical data and ontologies. We show the proposed method is indeed capable of capturing semantic (indirect) associations while seamlessly incorporate domain knowledge defined in formal ontologies. We also show that our data mining methods can discover misinformation in biomedical ontologies. Our work makes the following main contributions: First, we employ a RDF hypergraph representation to capture both semantics of ontologies and data. We can weight each hyperedge so that certain links (such as \emph{is\_a} or \emph{may\_treat} relationships) can carry appropriate strength.  Next, we serialize the hypergraph and weighted hyperedges into a bipartite representation for efficient processing. Then, we implement highly efficient and scalable random walks with restart over the bipartite graph to generate semantic associations, including associations that may not necessarily be co-frequent. The discovered semantic associations can be used to detect potential errors in biomedical ontologies.


\section{Related Work}
\subsection{Ontologies and Data Mining}
Using formal ontologies to annotate data has become increasingly popular in biomedical domains. For instance, in genetics, researchers curate literature to generate ontology-annotated data for different species of model organisms by linking specific proteins to various classes in the Gene Ontology (GO~\cite{GO}). These publicly available GO annotation databases make enrichment analysis possible, which enables researchers to functionally profile sets of interesting genes identified by microarray experiments~\cite{Khatri2005}.  At Stanford, the National Center for Biomedical Ontology (NCBO) annotates large volumes of biomedical text for search and mining~\cite{RI} and has been used, for instance, to profile disease research~\cite{Liu2012}.  Finally, millions of patient electronic health records are being annotated using medical ontologies like SNOMED-CT in efforts to advance patient healthcare~\cite{OMOP}.

In general domains, Staab and Hotho~\cite{StaabH03} were one of the earliest to utilize the idea of mapping terms in text to classes in an ontology and they essentially use the ontology to aggregate data and thus reduce feature dimensionality during clustering.  Adryan et al.~\cite{Adryan2004} enable cluster visualization for gene expression data by navigating various levels of Gene Ontology hierarchy.  %Shen et al.~\cite{Shen2006Ont} use the ontology to check for consistency of association rules. 
Wen et al.~\cite{Wen2007Ont} take into consideration the ontology hierarchy to offset biases toward overly-general terms in text mining.

\subsection{Graphs in Mining RDF and Ontologies}

RDF data and OWL ontologies can be represented as graphs for data mining. Lin et al.~\cite{Lin2011Learning} treat the RDF triple store as a datasource during mining and develop Relational Bayesian Classifiers (RBCs) that aggregate SPARQL queries. Kiefer et al.~\cite{Kiefer2008Adding} extend the SPARQL query language to enable creating and working with the data mining model. Similarly, Bicer et al.~\cite{Bicer2011Relational} define kernel machines over RDF data where features are constructed by ILP-based dynamic propositionalization. In each case, RDF is merely a data model, paying little attention to the use of domain-specific knowledge in related ontologies.

Recently, a promising graph-based approach to represent and mine ontologies and data together is Heterogeneous Information Networks (HIN) developed by Sun et al.~\cite{SunHYYW11, SunNHYYY12}. HIN leverages semantics of various types of nodes and links in a network for the graph and network mining tasks. Since ontologies and annotated data can be treated as a large graph of concepts and entities linked with different type of relationships, there should be a way to represent certain relationships and semantics (e.g., class subsumption) of ontologies and data in HIN. However, considering many biomedical ontologies are large, manually representing them in heterogeneous information networks is not practical. It is not easy to build an automatic translator from OWL ontologies to HIN either. We prefer RDF Hypergraphs because the RDF syntax of OWL ontologies and data makes it much easier to automatically transform existing biomedical ontologies and their annotated data into RDF hypergraphs. 