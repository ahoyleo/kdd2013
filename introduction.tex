\section{Introduction}
\label{sec:intro}

Researchers around the world are linking more and more data to ontologies which are formal specification of concepts and relationships in various domains. Formal ontologies have been extensively developed and harnessed in scientific research, particularly in biomedical research. Knowledge evolves rapidly in biomedicine and has promoted the creation and use of ontologies to advance scientific progress.  Besides the size of data increases exponentially, an increasing number of ``large" biomedical ontologies are developed. Prominent examples of this effort include the Gene Ontology (GO)~\cite{GO} and the Unified Medical Language System (UMLS)~\cite{UMLS}. Over 300 ontologies have been loaded into the National Center of Biomedical Ontology (NCBO) BioPortal library at Stanford~\cite{Noy2009}, specifying more than 5.6 million terms in the biomedical domain.

There are two major challenges facing researchers when it comes to mining extremely large sets of biomedical ontologies and data. The first is to leverage both the ontologies and the data in a systematic and scalable way. The second is to deal with errors in both ontologies and data since neither of them is perfect in reality. Previous research has not been sufficient to address these challenges. For example, some approaches have utilized ontologies in data mining, but the ontologies are typically partially used (mainly the subsumption relationship) on a specific portion of the task (most often concept aggregation). On the other hand, limited attempts have been made to check errors in large ontologies. Traditional approaches to sanitize knowledge bases by using an inference engine for consistency checking is hard to scale.

We believe that with the advent of increasing amount of ontology-annotated data, new possibilities are opened up for both data mining and ontology development. Therefore an emerging research direction, which we call \textit{semantic data mining}, focuses on drawing insights from both domain knowledge and data in a systematic way along with formal semantics defined in ontologies. It enables domain knowledge to be seamlessly brought into the data mining process, helping improve the quality of pattern discovered in a noisy environment. On the other hand, it also benefits the ontologies by having empirical substantiation from data to either bolster a priori ontological assertions, or detect potential errors therein.

Semantic data mining differs from other recent studies, by leveraging links between entities defined by ontologies---via annotations---to the mining algorithms explicitly in a unified model.  With ontology-annotated data, this would require traversing links across the ontologies to infer implicit inter-connections among the data. Graph techniques fit this research nicely because both domain knowledge (e.g., OWL~\cite{OWL} ontologies) and semantically rich data sets can be represented as graphs. For example, OWL is the standard ontology language built on RDF, therefore OWL and RDF documents are basically a set of RDF triples, which can be illustrated by a node-arc-node link. Therefore, any collection of RDF or OWL data is an RDF Graph~\cite{GraphModelRDF}. In fact, besides OWL and RDF, many semantically rich datasets of interest today are best described as a linked collection, or a graph, of interrelated objects~\cite{LinkMiningGetoor}. These graphs may represent homogeneous networks, in which there is a single object type and link type (such as web pages connected by hyperlinks), or richer, heterogeneous networks, in which there may be multiple object and link types (such as DBpedia, a data source containing structured information from Wikipedia).


Hence, our semantic data mining approach is inspired by a combination of graph-based representation~\cite{CheinMugnier08}, hypergraphs~\cite{Zhou06learningwith,Tian09}, and random walks~\cite{Fouss06random-walkcomputation,Pan,Yen05clusteringusing,Zhou:2009:GCB:1687627.1687709}. This paper extends our previous work that implements a hyper graph-based approach to learn associations from interlinked data (without ontologies)~\cite{LiuEtal11}. The hypergraph representation proposed by Hayes et al.~\cite{GraphModelRDF} is a key innovation. We adopt this representation in our approach because properties in RDF triples can be represented as first class objects among all interrelated objects, enabling us to embed both the ontology and the data together and to serialize it into a bipartite graph representation for scalable processing.

In RDF hypergraphs, we use random walk with restart to efficiently calculate the similarity between concepts so as to determine their associations from very large size of data and ontologies. Traditional association mining relies on co-frequencies of items (concepts) within transactions~\cite{Agrawal94}. We look one step further to find indirectly associated items or concepts. This simple extension has far reaching applications in biomedicine. For instance, consider a simple scenario illustrated by Swanson~\cite{swanson87} years ago while studying Raynauld's syndrome. He noticed that the literature discussed Raynauld's syndrome ($Z$), a peripheral circulatory disease, together with certain changes of blood in human body ($Y$); and, separately, the consumption of dietary fish oil ($X$) was also linked in the literature to similar blood changes.  But fish oil and Raynauld's syndrome were never linked directly in any previous publications.  Swanson reasoned (correctly) that fish oil could potentially be used to treat Raynauld's syndrome, i.e., $X\rightsquigarrow Y \rightsquigarrow Z$. We call such indirectly associated items, $(X,Z)$, \emph{semantic associations}.

We evaluate the effectiveness of the results on well-known shopping cart benchmark datasets, as well as the large real-world biomedical data and ontologies. We show the proposed method is indeed capable of capturing semantic (indirect) associations while seamlessly incorporate domain knowledge defined in ontologies through experiments performed with real life data and ontologies. We also show that our data mining methods can discover the errors in drug ontologies injected by human operations. Our work makes the following main contributions:  First, we employ a RDF hypergraph representation to capture both semantics of ontologies and data. We can weight each hyperedge so that certain links (such as \emph{is\_a} or \emph{may\_treat} relationships) can carry appropriate strength.  Next, we serialize the hypergraph and weighted hyperedges into a bipartite representation for efficient processing. Then, we implement highly efficient and scalable random walks with restart over the bipartite graph to generate semantic associations, including associations that may not necessarily be co-frequent. The discovered semantic associations can be used in turn to detect potential errors in biomedical ontologies.


\section{Related Work}
\subsection{Ontologies and Data Mining}
Using formal ontologies to annotate data has become increasingly popular in many scientific domains. For instance, in the genetic domain, researchers curate literature to generate ontology-annotated data for different species of model organisms by linking specific proteins to various classes in the Gene Ontology (GO). These publicly available GO annotation databases~\cite{GOAurl} make enrichment analysis possible, which enables researchers to functionally profile sets of interesting genes identified by microarray experiments~\cite{Khatri2005}.  At Stanford, the National Center for Biomedical Ontology (NCBO) annotates large volumes of biomedical text for search and mining, a technology which won the 2010 ISWC Semantic Web Challenge Open Track~\cite{RI,SWCurl} and has been used, for instance, to profile disease research~\cite{Liu2012}.  Finally, millions of patient electronic health records are being annotated using medical ontologies like SNOMED-CT and MedDRA in efforts to advance patient healthcare~\cite{OMOP}.

Staab and Hotho~\cite{StaabH03} were one of the earliest to utilize the idea of mapping terms in text to classes in an ontology and they essentially use the ontology to aggregate data and thus reduce feature dimensionality during clustering.  Adryan et al.~\cite{Adryan2004} enable cluster visualization for gene expression data by navigating various levels of Gene Ontology hierarchy.  Shen et al.~\cite{Shen2006Ont} use the ontology to check for consistency of association rules. Wen et al.~\cite{Wen2007Ont} take into consideration the ontology hierarchy to offset biases toward overly-general terms in text mining.

\subsection{Graphs in Data Mining with Ontologies}

Kiefer et al.~\cite{Kiefer2008Adding} extends the SPARQL query language to enable creating and working with the data mining model. Lin et al.~\cite{Lin2011Learning} also treats the RDF triple store as a datasource during mining and develops Relational Bayesian Classifiers (RBCs) that aggregate SPARQL queries. Similarly, Bicer et al.~\cite{Bicer2011Relational} define kernel machines over RDF data where features are constructed by ILP-based dynamic propositionalization. In each case, RDF is merely a data model, paying little attention to the use of domain-specific knowledge in related ontologies.

Recently, we notice that a promising graph-based approach to represent and mine the formal ontologies and data together is Heterogeneous Information Networks developed by Sun et al.~\cite{2012Sun, SunNHYYY12, YuSNMH12, YangCSH12, SunHYYW11}. Heterogeneous Information Networks leverages semantics of various types of nodes and links in a network for the graph and network mining tasks, such as rank-based clustering, classification, and similarity search. Since ontologies and annotated data can be treated as a large graph of concepts and entities linked with different type of relationships, there should be a way to represent certain relationships and semantics (e.g., class subsumption) of ontologies and data in Heterogeneous Information Networks. However, considering the size of biomedical ontologies are large, manually representing them in Heterogeneous Information Networks is not practical. There will be some overhead to build an automatic translator from OWL ontology to Heterogeneous Information Networks. The reason we choose RDF Hypergraphs is because that the RDF syntax of OWL ontologies and data make it much easier to transform existing biomedical OWL ontologies and their annotated data into RDF hypergraphs automatically.


