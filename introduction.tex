\section{Introduction}
\label{sec:intro}
Researchers around the world are linking more and more data to ontologies. There are two major challenges facing researchers when it comes to mining extremely large sets of ontologies and data. The first is to leverage both the ontologies and the data in a systematic and scalable way. The second is to deal with errors in both ontologies and data since neither of them is perfect in reality. Previous research has not been sufficient to address these challenges. For example, some approaches have utilized ontologies in data mining, but the ontologies are typically partially used (mainly the subsumption relationship) on a specific portion of the task (most often concept aggregation). On the other hand, limited attempts have been made to check errors in large ontologies. Traditional approaches to sanitize knowledge bases by using an inference engine for consistency checking is hard to scale. The justification in an inconsistent ontology is normally done by identifying ``minimal unsatisfiable subsets"~\cite{BaileyS05,HorridgePS09} of statements that cause it to be inconsistent. Deriving all minimal unsatisfiable sets will help generate the ``simplest" explanation (justification) for the errors.

We believe that with the advent of increasing amount of ontology-annotated data, new possibilities are opened up for both data mining and ontology development. Therefore our new method, which we call \textit{semantic data mining}, focuses on drawing insights from both domain knowledge and data in a systematic way along all kinds of relationships defined by the ontology. It enables domain knowledge to be seamlessly brought into the data mining process, helping improve the quality of pattern discovered in a noisy environment. On the other hand, it also benefits the ontologies by having empirical substantiation from data to either bolster a priori ontological assertions, or detect potential errors therein.


Semantic data mining differs from other recent studies, by leveraging links between entities defined by ontologies---via annotations---to the mining algorithms explicitly in a unified model.  With ontology-annotated data, this would require traversing links across the ontologies to infer implicit inter-connections among the data. Hence, semantic data mining is inspired by a combination of graph-based representation~\cite{CheinMugnier08}, hypergraphs~\cite{Zhou06learningwith,Tian09}, and random walks~\cite{Fouss06random-walkcomputation,Pan,Yen05clusteringusing,Zhou:2009:GCB:1687627.1687709}. This work extends our previous work that implements a hyper graph-based approach to learn associations from interlinked data (without ontologies)~\cite{LiuEtal11}. The hypergraph representation proposed by Hayes et al.~\cite{GraphModelRDF} is a key innovation. We adopt this representation in our approach because properties in RDF triples can be represented as first class objects among all interrelated objects, enabling us to embed both the ontology and the data together and to serialize it into a bipartite graph representation for scalable processing.

The mining problem we aim to solve is an extension to one of the most basic and useful data mining tasks, association rule mining~\cite{Agrawal94}. Traditional association rule mining relies on co-frequencies within transactions. We look one step further to find indirectly associated items. This simple extension has far reaching applications in healthcare. For instance, consider a simple scenario illustrated by Swanson~\cite{swanson87} years ago while studying Raynauld's syndrome. He noticed that the literature discussed Raynauld's syndrome ($Z$), a peripheral circulatory disease, together with certain changes of blood in human body ($Y$); and, separately, the consumption of dietary fish oil ($X$) was also linked in the literature to similar blood changes.  But fish oil and Raynauld's syndrome were never linked directly in any previous studies.  Swanson reasoned (correctly) that fish oil could potentially be used to treat Raynauld's syndrome, i.e., $X\rightsquigarrow Y \rightsquigarrow Z$. We call such indirectly associated items, $(X,Z)$, \emph{semantically associated itemsets}. 

Our work makes the following main contributions:  First, we employ a hypergraph representation to capture both ontologies and data. We can weight each hyperedge so that certain links (such as \emph{is\_a} or \emph{may\_treat} relationships) can carry appropriate strength.  Next, we serialize the hypergraph and weighted hyperedges into a bipartite representation for efficient processing. Then, we implement highly efficient and scalable random walks with restart over the bipartite graph to generate frequent itemsets, including associations that may not necessarily be co-frequent. The discovered semantic associations can be used in turn to detect potential errors in ontologies. Finally, we evaluate the effectiveness of the results on well-known shopping cart benchmark datasets, as well as the large real-world healthcare dataset.

\section{Related Work}
\subsection{Ontology Annotated Datasets}
Using formal ontologies to annotate data has become increasingly popular in many scientific domains. For instance, in the genetic domain, researchers curate literature to generate ontology-annotated data for different species of model organisms by linking specific proteins to various classes in the Gene Ontology (GO). These publicly available GO annotation databases~\cite{GOAurl} make enrichment analysis possible, which enables researchers to functionally profile sets of interesting genes identified by microarray experiments~\cite{Khatri2005}.  At Stanford, the National Center for Biomedical Ontology (NCBO) annotates large volumes of biomedical text for search and mining, a technology which won the 2010 ISWC Semantic Web Challenge Open Track~\cite{RI,SWCurl} and has been used, for instance, to profile disease research~\cite{Liu2012}.  Finally, millions of patient electronic health records are being annotated using medical ontologies like SNOMED-CT and MedDRA in efforts to advance patient healthcare~\cite{OMOP}.

\subsection{Ontologies in Data Mining}
Staab and Hotho~\cite{StaabH03} were one of the earliest to utilize the idea of mapping terms in text to classes in an ontology and they essentially use the ontology to aggregate data and thus reduce feature dimensionality during clustering.  Adryan et al.~\cite{Adryan2004} enable cluster visualization for gene expression data by navigating various levels of Gene Ontology hierarchy.  Shen et al.~\cite{Shen2006Ont} use the ontology to check for consistency of association rules. Wen et al.~\cite{Wen2007Ont} take into consideration the ontology hierarchy to offset biases toward overly-general terms in text mining. 

\subsection{Graphs in Data Mining with Ontologies}
Kiefer et al~\cite{Kiefer2008Adding} extends the SPARQL query language to enable creating and working with the data mining model. Lin et al.~\cite{Lin2011Learning} also treats the RDF triple store as a datasource during mining and develops Relational Bayesian Classifiers (RBCs) that aggregate SPARQL queries. Similarly, Bicer et al~\cite{Bicer2011Relational} define kernel machines over RDF data where features are constructed by ILP-based dynamic propositionalization. In each case, RDF is merely a data model, paying little attention to the use of domain-specific knowledge in related ontologies. 