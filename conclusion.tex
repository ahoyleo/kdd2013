\section{Conclusion and Future Work}
\label{sec:conclusion} 
We approach the discovery of indirectly associated items from ontology-annotated data, called \emph{semantically associated itemsets}, by proposing a new semantic data mining technique.  Our method uses hypergraphs and random walks with restart over a bipartite graph serialization to discover associations that cannot be found by methods that rely on co-frequent items because it utilizes both the ontology and the data at the same time.  Moreover, we allow users the ability to customize the weight of each component, giving flexibility in how strongly the role of the ontology plays over the data, or vice-versa.  Our evaluations show that the method discovers indirect associations and that it scales to datasets that are large in multiple says: the ontology can be large and the data can be large.

In future work, we will explore algorithms that suggest appropriate weights to apply to the components of the hypergraph. Also, we will implement methods for clustering and classification tasks within this framework. We will focus on our healthcare datasets mainly because they are both large and complex, but also because of the enormous potential in advancing the state-of-the-art in clinical informatics and improving the quality of care for millions of patients.