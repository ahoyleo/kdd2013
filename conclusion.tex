\section{Conclusion and Future Work}
\label{sec:conclusion}

%We approach the discovery of indirectly associated items from ontology-annotated data, called \emph{semantically associated itemsets}, by proposing a new semantic data mining technique.  Our method uses hypergraphs and random walks with restart over a bipartite graph serialization to discover associations that cannot be found by methods that rely on co-frequent items because it utilizes both the ontology and the data at the same time.  Moreover, we allow users the ability to customize the weight of each component, giving flexibility in how strongly the role of the ontology plays over the data, or vice-versa.  Our evaluations show that the method discovers indirect associations and that it scales to datasets that are large in multiple says: the ontology can be large and the data can be large.
%
%In future work, we will explore algorithms that suggest appropriate weights to apply to the components of the hypergraph. Also, we will implement methods for clustering and classification tasks within this framework. We will focus on our healthcare datasets mainly because they are both large and complex, but also because of the enormous potential in advancing the state-of-the-art in clinical informatics and improving the quality of care for millions of patients.

We mine biomedical ontologies and  data using a unified RDF hypergraph representation.  Our method uses random walks with restart over a bipartite graph serialization to discover semantic associations that cannot be found by methods that rely on co-frequencies. We are one of the first research groups to consider mining biomedical ontologies and data together without using a separate system for pre-processing or pre-computing ontologies. We allow users to customize the weight of each semantic component, providing flexibility to express how strongly the role of the ontology plays over the data, or vice-versa. Our evaluations show that the method discovers semantic associations and that it scales to to both the size of data and the size of ontologies. Moreover, we also show that our methods can discover and suggest corrections for misinformation in biomedical ontologies.

In the following we discuss some future research directions in mining the unified RDF bipartite graph.

Developing scalable semantic data mining algorithms is critically important. The electronic health dataset in our study has grown beyond 100 billion triples and the size of the ontologies also tremendous (SNOMED-CT has nearly 400,000 classes). The RWR method works well for query-based node similarity, but it is not applicable to generate full pair-wise node similarities at such scale. While the size of practical problem is bound to increase, the graph-based formalism of our method makes it possible to leverage decades of work on graph mining. A promising direction going forward is to employ approximation and develop parallelizable algorithms.

The weighted hyperedges provide a great deal of flexibility to users who may prefer domain knowledge over data (or vice versa) and opens up new research questions on how to optimally configure learning algorithms. In reality, the appropriate ratio for the edge weights is not only dependent on the size of graphs but also the graph configuration (depth, average degree, etc). Moreover, allowing users to specify the ratio of prior knowledge in ontologies versus inductive evidence from data enables us to discover empirically optimal configurations. We would explore a few permutations on these hyperedge weights going forward.

The RDF bipartite graph representation has limited expressivity compared to OWL itself. For example, domain, range and cardinality constraints are not straightforward to model.  One possible approach is to model domain constraints by explicitly describing the desired or acceptable walk (traversal sequence) in the RDF hypergraph. In this case, the recently proposed \emph{regular traversal expression} ~\cite{Marko10} technique may apply. However, their fast power-iteration approach for computing the stationary probability may not be applicable any more due to the label sequence constraint, but the Monte-Carlo simulation of the random walk may help to approximate the similarity measure.

We will continue to focus on healthcare datasets mainly because large number of biomedical ontologies have been developed and used to annotate the real-life data, but also because of the enormous potential in advancing the state-of-the-art in clinical informatics and improving the quality of care for millions of patients. This research also contributes to an emerging research direction of semantic data mining, in which formal semantics that exist in data and knowledge can be represented and incorporated into the data mining process in a seamless way.



