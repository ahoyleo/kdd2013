\section{Conclusion and Future Work}
\label{sec:conclusion}

%We approach the discovery of indirectly associated items from ontology-annotated data, called \emph{semantically associated itemsets}, by proposing a new semantic data mining technique.  Our method uses hypergraphs and random walks with restart over a bipartite graph serialization to discover associations that cannot be found by methods that rely on co-frequent items because it utilizes both the ontology and the data at the same time.  Moreover, we allow users the ability to customize the weight of each component, giving flexibility in how strongly the role of the ontology plays over the data, or vice-versa.  Our evaluations show that the method discovers indirect associations and that it scales to datasets that are large in multiple says: the ontology can be large and the data can be large.
%
%In future work, we will explore algorithms that suggest appropriate weights to apply to the components of the hypergraph. Also, we will implement methods for clustering and classification tasks within this framework. We will focus on our healthcare datasets mainly because they are both large and complex, but also because of the enormous potential in advancing the state-of-the-art in clinical informatics and improving the quality of care for millions of patients.

We mine biomedical ontologies and  data using a unified RDF hypergraph representation.  Our method uses random walks with restart over a bipartite graph serialization to discover semantic (indirect) associations that cannot be found by methods that rely on co-frequencies because our method utilizes both the ontology and the data at the same time. We are one of the first research groups which consider to mine biomedical ontologies and data together without using a separate system to pre-processing or pre-computing ontologies. We allow users the ability to customize the weight of each component, giving flexibility in how strongly the role of the ontology plays over the data, or vice-versa.  Our evaluations show that the method discovers semantic associations and that it scales to datasets that are large in multiple says: the ontology can be large and the data can be large. Moreover, we also show that our data mining methods can discover the errors in biomedical ontologies injected by human operations.

In future work, we will explore algorithms that suggest appropriate weights to apply to the components of the hypergraphs. Also, we will implement methods for clustering and classification tasks within this framework. We will focus on healthcare datasets mainly because large number of biomedical ontologies have been developed and used to annotate the real-life data, but also because of the enormous potential in advancing the state-of-the-art in clinical informatics and improving the quality of care for millions of patients. This research also contributes to an emerging research direction of semantic data mining, in which formal semantics that exist in data and knowledge can be represented and incorporated into the data mining process in a seamless way.



