\section{Related Work}
\label{sec:related}
In this section, we discusses the current state of the effort to incorporate domain knowledge especially formal ontologies in data mining. In addition, we also describe the various formalisms proposed in the literature to use graphs in tackling Semantic Web and data mining problems.

%\subsection{Ontologies in Data Mining}
%Domain knowledge relates to information about a specific domain or data that is collected from previous systems or documentation, or elicited from domain experts.

%We first highlight a selection of studies that aims at exploring domain knowledge encoded in formal ontologies in data mining. Domain knowledge can affect data mining systems in at least two ways. First, it makes patterns more visible by generalizing attribute values; and second, it often reduces the search space by introducing constraints and heuristics.

%Staab and Hotho~\cite{StaabH03} describe an ontology-based text clustering approach. They develop a preprocessing method, called COSA, one of the earliest to utilize the idea of mapping terms in the text to concept in the ontology. By mapping terms to concepts, it essentially aggregates terms and reduces the dimensionality. Adryan et al.~\cite{Adryan2004} developed a system called GO-Cluster which uses the tree structure of the Gene Ontology database as a framework for numerical clustering, and thus allowing a simple visualization of gene expression data at various levels of the ontology tree. Shen et al.~\cite{Shen2006Ont} proposed a method of association rules retrieval that is based on ontology and Semantic Web. Ontological semantics and reasoning can be used for sharing and consistency checking of discovered association rules. Wen et al.~\cite{Wen2007Ont} propose a technique for concept frequency re-weighing to solve the problem in text mining that a document is often biased towards class-independent ``general" words while short of class-specific ``core" words. The proposed method takes into consideration the concept hierarchy defined in the domain ontology.

More recent studies have focused on developing scalable and incremental data mining systems by tapping on the distributed characteristics of OWL and RDF. Kiefer et al~\cite{Kiefer2008Adding} propose to perform statistical learning and inference on RDF data by directly extending the SPARQL query language that enables creating and working with a data mining model in SPARQL. Lin et al.~\cite{Lin2011Learning} describe an approach to learning Relational Bayesian Classifiers (RBCs) from RDF data by querying statistics of data through the SPARQL endpoint thus allowing the storage to be decentralized. They also establish the conditions under which the RBC models can be incrementally learned in response to changes of RDF data.

%However, previous ontology-based data mining approaches treat the data and ontologies separately and handle the tasks in much smaller scale especially respect to the number ontological concepts in real life ontology-annotated data we deal with in this paper.


\subsection{Graph-based formalism in Semantic Web and Data Mining}

Given the fact that OWL ontologies and RDF documents in Semantic Web have a graph nature, we hope to employ a graph-based formalism to represent annotated data so that the ontologies and data can be combined for mining in a systematic manner. Hayes has established that RDF can be also represented as hypergraphs (bipartite graphs)~\cite{GraphModelRDF}. The closely related ones to this paper are random work and hypergraphs in a number of data mining systems.
%Various quantities derived from random walk has been proposed to model the relevance, or similarity between entities in a graph. Fouss et al.~\cite{Fouss06random-walkcomputation} compare twelve scoring algorithms based on graph representation of the database to perform collaborative movie recommendation. Pan et al.~\cite{Pan} develop a similarity measure based on random walk steady state probability to discover correlation between multimedia objects containing data of various modalities.
Yen et al.~\cite{Yen05clusteringusing} introduce a new k-means clustering algorithm utilizing the random walk average commute time distance. Zhou et al.~\cite{Zhou:2009:GCB:1687627.1687709} presented a unified framework based on neighborhood random walk to integrate structural and attribute similarities for graph clustering. The concept of random walk is extended to hypergraph~\cite{Zhou06learningwith}. Tian et al.~\cite{Tian09} propose a hypergraph-based learning algorithm to classify gene expression data. They also take into account previous knowledge by modeling the knowledge as regularization terms in the optimization problem.
In our previous work, the authors have developed a hypergraph-based method for discovering semantically associated itemsets~\cite{LiuEtal11}. Although the results are promising and interesting to domain experts, we have only utilized the data graphs for association mining but not the rich semantics of ontologies. In this paper,  we presents a Semantic Data Mining framework which will draw insights from both ontology and data in a systematic and efficient way.

