\section{Discussion}
\label{sec:discussion}
We have demonstrated that using the proposed combined RDF bipartite graph incorporates both domain knowledge and data based on the user's desired weights for each component for finding \emph{semantically associated itemsets}.

Developing scalable algorithms for semantic data mining is critically important. In our work, the healthcare dataset has grown beyond 100 billion triples and the size of the ontologies used are also large (SNOMED-CT has nearly 400,000 classes). The RWR method we describe works well for query-based node similarity, but it will not scale to generate full pair-wise node similarities at this tremendous scale because a calculation of eigenvectors of the Laplacian is required to derive the similarity measures, which is very expensive on large graphs.

While non-trivial practical problems associated with very large graphs cannot be completely avoided, our work makes it possible to leverage decades of work on graph-based methods in this effort to mine semantically associated data. The general strategy going forward is to employ approximation and develop parallelizable algorithms. Lin and Cohen~\cite{LinEtal2010ICML} proposed an approximation to a eigenvalue-weighted linear combination of all the eigenvectors, which can be achieved by performing a small number of matrix-vector multiplications.  The procedure results in a more scalable method called \emph{power iteration clustering} that finds a very low-dimensional data embedding using truncated power iteration on a normalized pair-wise similarity matrix of the data points. Zhao et al.~\cite{ZhaoEtal2011Eff} described the idea of \emph{graph coordinate systems}, which provides a fast embedding of large graphs into a hyperbolic space. The method parallelizes easily and efficiently locate shortest paths between node pairs, which relates well to the notion of commute time distance, which our RWR method seeks to elicit. Savas and Dhillon~\cite{SavasEtal2011Clu} introduced a novel framework called \emph{clustered low rank matrix approximation for massive graphs}. After a few intermediate steps, they are able to finally project an a optimal, low rank approximation of the entire graph, thus including connections or edges between vertices from different clusters. We intend to extend these ideas to ontology-annotated hypergraphs.

The weighted hyperedges provide a great deal of flexibility to users who may prefer domain knowledge over data (or vice versa) and opens up new research questions on how to optimally configure learning algorithms. In reality, the appropriate ratio for the edge weights is not only dependent on the size of graphs but also the graph configuration (depth, average degree, etc). Moreover, allowing users to specify the ratio of prior knowledge in ontologies versus inductive evidence from data enables us to discover empirically optimal configurations. We have performed exhaustive feature selection on classification algorithms for our healthcare dataset in the past, and we would also explore a few permutations on these hyperedge weights going forward. We can draw upon other works, such as, Tian et al.~\cite{Tian2009AHyper}, who proposed a semi-supervised approach for classifying nodes in a graph based on a relatively small labeled set.

The RDF bipartite graph representation has limited expressivity compared to OWL itself. For example, domain, range and cardinality constraints are not straightforward to model.  One possible approach is to model domain constraints by explicitly describing the desired or acceptable walk (traversal sequence) in the RDF hypergraph. In this case, the recently proposed \emph{regular traversal expression} ~\cite{Marko10} technique may apply. However, their fast power-iteration approach for computing the stationary probability may not be applicable any more due to the label sequence constraint, but the Monte-Carlo simulation of the random walk may help to approximate the similarity measure.

We have showcased the utility of the RDF bipartite graph in mining \emph{semantically associated itemsets} and will explore other data mining tasks as well. For example, the the classification task can be formulated as discovering the correct labels for unlabeled vertices in the weighted bipartite graph. Conceptually, the pair of nodes being ``close to'' one another shall share the same labels, which are consistent to one basic principle of semi-supervised learning, so the challenge is to define the closeness between two nodes. The clustering task can be viewed as a neighborhood formation for vertices on the data partition. Basically, for any closely related nodes, we group them together. Again, with an explicit similarity measure, possibly even the classical k-means in this case, could be directly applied to cluster the vertices. 